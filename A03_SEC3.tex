
\subsection{A* Search Algorithm}
A* is a computer algorithm which is widely used to solve graph search problems \cite{a8book}. It determines the most efficient path between multiple nodes in a graph. A* is an informed search algorithm. This means it searches between all the possible paths to the goal and finds the path incurring the least cost. To do this, it considers the paths which appear to have the least cost to get to the goal first. Starting from a specific node, it explores the graph step by step depending on the cost of going from one node to the next. The algorithm selects the node to explore based on a combination of the actual cost to get to the node and a heuristic cost that estimates the cost from the current node to the goal. The process to calculate the heuristic cost is problem-specific. For the algorithm to work correctly, the heuristic cost has to be less than or equal to the actual cost of getting to the goal node. In other words, the heuristic cost function should never overestimate the cost of reaching the goal. The algorithm works by calculating the combined actual and heuristic cost for all the neighboring nodes of the starting node and puts them into a priority queue called the \textit{open list}. Then, it selects the node with the least cost and expands that node to get the cost of its neighbors. The expanded node is taken out of the priority queue and put in another list called the \textit{closed list}. The algorithm continues to expand the \textit{open list} always selecting the node with the least cost and adding it to the \textit{closed list}. The algorithm terminates when the goal node is inside the \textit{open list}, and it has the minimum cost in the \textit{open list}. Finally, the algorithm retraces its path through the \textit{closed list} to find the optimum route from start to goal. Pseudocode for the A* algorithm is given below.

\textbf{Algorithm 1:} A* search algorithm

\begin{algorithmic}[1]
\label{al:1}
% \Function{heuristic\_estimate}{$a,b$}
%     \State $hCost := \sum_{n=a}^b D(n)*R_{best}(n) $
%     \State \Return $hCost$
% \EndFunction

% \Function{gCost\_calc}{$p,c$}
%     \State $C_{actual}(pc) := C_{ESS}(pc)+C_{GRID}(t)+C_{best}(p)$
%     \State \Return $C_{actual}(pc)$
% \EndFunction

\Function{A*}{$start, goal$}
\State $Closed\_Set = \{\}$ \Comment{Set of evaluated nodes}
\State $Open\_Set = \{Start\}$ \Comment{Set of already discovered nodes which have not been evaluated. Initially,  the start node is discovered.}
\State $F\_cost[Start] = \infty$
\While{$Open\_Set$ is not empty}
    \State $Current\_Node = $ node in $Open\_Set$ with lowest $F\_cost$
    \If{$Current\_Node$ == goal}
      \State  \textbf{break} 
    \EndIf
    \For {Each child node of $Current\_Node$}
        \If{Child node in $Closed\_Set$}
            \State \textbf{continue}
        \EndIf
        \If{Child node in $Open\_Set$}
            \If{$G\_cost[child]$ $\geq$ $G\_cost$ already in $Open\_Set$}
                \State \textbf{continue}
            \EndIf
        \EndIf
        \State $Best\_Parent[child] = Current\_Node$
        \State $F\_cost[child] = G\_cost[child] + H\_cost[child]$
        \State $Open\_Set.add(child)$ \Comment{Add child node to $Open\_Set$}

    \EndFor
    \State $Closed\_Set.add(Current\_Node)$ \Comment{Add $Current\_Node$ to $Closed\_Set$}
    \State $Open\_Set.remove(Current\_Node)$ \Comment{Remove $Current\_Node$ From $Open\_Set$}
\EndWhile
\State $Best\_Path = \{ \}$ \Comment{$Best\_Path$ is the most efficient path to get to the goal node.}
\While{$Best\_Parent[Current\_Node] != Start$}
\State $Current\_Node = Best\_Parent[Current\_Node]$
\State $Best\_Path.add(Current\_Node)$ \Comment{Add $Current\_Node$ to $Best\_Path$}
\EndWhile
\State \Return $Best\_Path$
\EndFunction
\end{algorithmic}

The steps of the algorithm and how it works is explained with a simple example in Fig. \ref{fig:A_STAR_EXAMPLE}. The nodes with a white background are the unexplored nodes. The Nodes with the orange background are the nodes in the open list and nodes with a gray background are the nodes in the closed list. The black arrows show the edges of the graph and the numbers on them represent the cost of taking the edge. The dotted red lines represent heuristic costs from a node to the goal node and their values are represented by the red numbers. The graph shown in the figure is represented as
\begin{equation}
    Z = ( V(Z), X(Z) )
\end{equation}
Here, $V$ is the set of nodes or vertices and $X$ is the set of ordered pairs or edges used the represent the graph Z. Their elements are shown in (\ref{eq:V_Z}) and (\ref{eq:X_Z}) respectively.
\begin{equation}
\label{eq:V_Z}
    V(Z) = \{A,B,C,E,F,G,H\}
\end{equation}
\begin{equation}
\label{eq:X_Z}
    X(Z) = \{ (A,B), (B,E), (E,H), (A,C), (C,G), (C,F), (F,H) \}
\end{equation}

The goal here is to start at node A of Z and find the shorted path to reach node H. In the beginning, Algorithm 1 is given the start node A and goal node H. Then at line 2 the algorithm initializes an empty $Closed\_Set$. Then it initializes the $Open\_Set$ with the starting node, which is A in line 3. Then the $F\_cost$ of node A is set to $\infty$ at line 4. These steps are represented in Fig.\ref{fig:A_STAR_EXAMPLE}(a).  Then we enter the while loop in line 5 of the algorithm. As the $Open\_Set$ contains the node A, the loop continues. On line 6, $Current\_node$ is set to A as A is the only node in the $Open\_Set$. Now, as $Current\_node$ is not H (goal node), we continue to line 10 of the algorithm. Here, for each child node of $Current\_node$, the algorithm first checks if the node is in the $Closed\_Set$. If the child node is in $Closed\_set$, the child node is ignored. Then on lines 14 \& 15 the algorithm checks if the child node is in $Open\_set$ and is the $G\_cost$ of getting to the child is higher than the $G\_cost$ currently present in $Open\_Set$ then the child is ignored. Otherwise, we continue with the algorithm and in line 19 the $Current\_Node$ is set as the best parent of the child node. Then the $F\_cost$ of the child node is calculated by adding the $G\_cost$ and $H\_cost$ of the child in line 20. Then the child is added to the $Open\_set$ in line 21. After that in line 23 and 24 of the algorithm, the current node is added to the $Closed\_Set$ and taken out of the $Open\_Set$. In the example shown in Fig. \ref{fig:A_STAR_EXAMPLE}, the two children B \& C of node A are explored in this step. As none of them are in $Open\_Set$ or $Closed\_Set$ we calculate their F costs and add them to the $Open\_Set$. Also, node A is taken out of the $Open\_Set$ and added to the $Closed\_Set$. These changes can be noticed in Fig. \ref{fig:A_STAR_EXAMPLE}(b). The gray background behind node A represents that it is in the $Closed\_Set$ and the orange backgrounds behind nodes B and C represent they are in the $Open\_Set$. Now the algorithm goes back to line 5 and as $Open\_Set$ is not empty it continues on. This time the node with the lowest cost is B, so the children of B are explored. Which reveals the new node E, which is added to the $Open\_Set$ and B is taken out of the $Open\_Set$ and added to the $Closed\_Set$. This brings us to the state of the search shown in Fig. \ref{fig:A_STAR_EXAMPLE}(c). $Open\_Set$ not empty so the algorithm continues and chooses node E as current node as it has the lowest $F_Cost$ in the $Open\_Set$. This leads to the discovery of node H, which is our goal node. H is added to the $Open\_Set$. E is taken out of the $Open\_Set$ and added to the $Closed\_Set$. These steps are reflected in Fig. \ref{fig:A_STAR_EXAMPLE}(d). Now we have a path to H, and H is the node with the smallest F cost in the $Open\_Set$. SO the algorithm exits the while loop on line 8 and continues to line 26. here best path is initialized as an empty set. Then the algorithm traces back the $Best\_Parent$ of the $Current\_node$ in the while loop between lines 27 to 30 in the algorithm until it reaches the start node. And that gives us the $Best\_Path$. This is shown in Fig. \ref{fig:A_STAR_EXAMPLE_2}. In this case, the best path from A to H is A, B, E \& H. This path is shown by the nodes highlighted in blues in the figure.

\begin{figure}[!ht]
    \centering
    \includegraphics[width = \linewidth]{figs/A8/A_STAR_EXAMPLE.png}
    \caption{A* path finding example to find the shortest path between the nodes A and H of graph Z.}
    \label{fig:A_STAR_EXAMPLE}
\end{figure}


\begin{figure}[!ht]
    \centering
    \includegraphics[width = 0.5\linewidth]{figs/A8/A_STAR_EXAMPLE_2.png}
    \caption{A* path finding example best path from A to H.}
    \label{fig:A_STAR_EXAMPLE_2}
\end{figure}


% Fig. \ref{fig:A_STAR_PIC} shows an example of the A* algorithm in a simple directional graph. The capital letters A, B, C, E, F, G and H inside the ellipses represent the nodes of the graph. The starting point of the graph is A, and the goal is to find the shortest path to node H. The solid blue arrows indicate the directional path from one node to the next. The actual cost of taking these paths are represented by the blue numbers beside the solid blue arrows. The red dotted lines represent the heuristic costs from a node to the target (goal) node. The heuristic costs of nodes B, C and G are shown as 7, 6 and 8. The letter 'h' is used to denote this cost. The algorithm starts at node A and discovers the surrounding nodes B and C. The combined heuristic and actual cost of node B is 9 and it is denoted by the letter 'f'. The 'f' cost for node C is 10. So, the algorithm will choose node B as the main candidate to explore. In node B, node E is discovered to have a combined cost of 10. Now, between the nodes C and E, the algorithm will choose a node at random. In this case, let us assume the node selected by the algorithm is E. After expanding E, the algorithm will find a path to node H, which costs 10. However, this is still not the lowest path cost among all the nodes in the priority queue. C also has a cost of 10. So, the algorithm will expand the node C and discover two new nodes G and F. The combined heuristic and actual cost of node G is 15 and node F is 14, which in turn are more than 10. So, the algorithm will stop searching and decide that the shortest path from node A to H is the path previously discovered. By using this approach, A* lets us avoid exploring the nodes G and F because the combined heuristic and actual cost of a node are less than or equal to the actual cost taking a path through that node to the target.





% \begin{figure}[!ht]
% \centering
% %\includegraphics[width=\linewidth]{figs/A_STAR_PIC.png}
% \includegraphics[width=\linewidth]{figs/A_STAR_PIC.png}
% \caption{Example graph for A* implementation}
% \label{fig:A_STAR_PIC}
% \end{figure}

\subsection{Implementation}
By defining the solution space with a combination of nodes and edges, the ESM optimization problem can be formulated as a graph search problem as well. At the inception, the starting node is determined by the current status of the system. Then, the following nodes and edges are generated using the forecasted data available. A discrete set of endpoints are set as the goals of the search. The A* algorithm stops when one of the endpoints are in the $Open\_Set$ and they have the lowest $F\_cost$. The cost of going from a parent node to a child node is calculated by combining the real cost of getting to that child node and the heuristic cost of getting to the goal from that child node. The real cost of going from a parent node $p$ at time $T=t$ to a child node $c$ at time $T=t+\Delta T$ is denoted as $C_{actual}(pc)$. It is calculated according to (\ref{eq:C_actual}).

\begin{equation}
\label{eq:C_actual}
    C_{actual}(pc) =  C_{ESS}(pc)+C_{GRID}(t)+C_{best}(p)
\end{equation}

Here, $\Delta T$ represents the time between two time steps. $C_{actual}(pc)$ represent the total cost of going to the child node $c$ from parent node $p$. $C_{ESS}(pc)$ represent cost of energy storage to go from parent node $p$ to child node $c$. $C_{GRID}(t)$ is the cost of using the grid between time $T=t$ and time $T=t+\Delta T$. $C_{best}(p)$ represent the best or least cost to get to the parent node $p$ from the start node. $C_{ESS}(pc)$ is calculated according to (\ref{eq:C_ESS}).

\begin{equation}
\label{eq:C_ESS}
C_{ESS}(pc) = |(SOC_p - SOC_c)|*ESS_{CAP}*R_{ESS} 
\end{equation}

Here, $SOC_p$ and $SOC_c$ represent the state of charge at parent and child node. $ESS_{CAP}$ represent the total energy capacity of the energy storage. And $R_{ESS}$ is the $\$/kWh$ cost of using the energy storage. $C_{GRID}(t)$ is calculated according to (\ref{eq:C_GRID}).

\begin{equation}
\label{eq:C_GRID}
C_{GRID}(t) = 
\begin{cases}
   E_{GRID}(t)*RTP(t),& \text{if } E_{GRID}(t)\geq 0\\
    E_{GRID}(t)*SP(t),& \text{if }  E_{GRID}(t) < 0
\end{cases}
\end{equation}

Here, $E_{GRID}(t)$ is the energy drawn from the grid between time $T=t$ and time $T=t+\Delta T$. $RTP(t)$ is the real-time price between $t$ and $t+\Delta T$. $SP(t)$ is the price the utility is willing to pay the consumer for selling power between $t$ and $t+\Delta T$. The heuristic cost is calculated by assuming that whichever source has the smallest cost during a time step will supply the total energy demand of that particular time step. The heuristic cost of a node at time $T = t$ is calculated according to (\ref{eq:C_H}).

\begin{equation}
\label{eq:C_H}
C_H(t) = \sum_{n=t}^{end} D(n)*R_{best}(n)
\end{equation}

Here, $C_H(t)$ represents the heuristic cost at time $t$. $D(n)$ is the demand between time $T = n$ and time $T = n+\Delta T$. $R_{best}(n)$ is the source with the smaller cost which is calculated according to (\ref{eq:R_best}).

\begin{equation}
\label{eq:R_best}
R_{best}(n) = 
\begin{cases}
    R_{ESS},& \text{if } RTP(n)\geq R_{ESS}\\
    RTP(n),              & \text{otherwise}
\end{cases}
\end{equation}

After calculating the actual cost  $C_{actual}(pc)$ and heuristic cost $C_H(t)$, the total cost is finally calculated by adding  $C_{actual}(pc)$ and $C_H(t)$.

The EMS recalculates the optimum path using the search algorithm at every time step based on updated information. The system status is assumed to be constant between time steps. 