\section{Protection}
The traditional protection devices available in the distribution network are circuit breakers, reclosers, sectionalizes and fuses. The traditional grid protection is coordinated using inverse definite minimum time (IDMT) curves. These curves are used to rate the protection devices to protect against over current and earth faults. The IDMT based solutions were optimized to make sure minimum number of customers were affected by the faults. The traditional network had only one source of fault current. So the IDMT based solutions did not take into account the effects of DERs. These protection schemes have been shown to be insufficient with a high DG penetration in \cite{PR133} and \cite{PR134}. The IEEE Standard 1547-2003 states that
“Any distributed resource installation connected to a spot network shall not cause operation or prevent reclosing of any network protectors installed on the spot network. This coordination shall be accomplished without requiring any changes to prevailing network protector clearing time practices of the area electric power system \cite{PR135}.” 
So according to the current standards there should be a maximum limit to DG integration in the system. There is research going on to determine the specific boundary of DG integration without altering the traditional protection scheme \cite{PR136}. 
From the research presented in \cite{PR137} it is evident that if we want to increase DG penetration in the distribution system, we need to change our protection system to counter the effects of DG integration. One existing protection scheme that may be implemented to tackle these issues is the over current (OC) protection relay with directional elements. This technology is quite mature as has been used in the traditional distribution system \cite{PR137}. Unlike the traditional distribution system, the transmission system always had multiple sources feeding into it. And in theory, the directional OC protection scheme implemented in transmission systems can be a viable solution for the distribution grid. This method of protection is proposed in \cite{PR138}. The authors of  \cite{PR138} propose that  a DER connected in the middle of a distribution line should have directional protection devices on each side of the point of common coupling (PCC). There are many drawbacks to this proposal. Firstly, the method proposed will require two directional protection relays for each DER connected to the grid. It would be very costly. They also assume that the DERs are interfaced through synchronous units. The DERs interfaced with power electronics may not produce enough fault current to trip the protection system.  Moreover, the non-directional protection connected at the beginning of the feeder may trip due to fault in a neighboring feeder  \cite{PR137}.
 In \cite{PR139} Choi J. et al. propose an adaptive protection scheme. The proposed protection scheme used OC and voltage based methods for fault detection. Threshold values of ‘critical current’ and ‘critical voltage’ are defined for PCC of the Dg units. These critical values are used in case of faults within the inter-tie circuit breaker zone. If the relay detects these critical values within the inter-tie circuit breaker zone, an instantaneous trip occurs. Otherwise, the IDMT curve based protection scheme is used. The protection scheme proposed can be inadequate for high impedance faults caused by inverter interfaced DGs. The distribution lines may cover large areas and substation voltage drops may be frequent \cite{PR137}. This would likely cause the voltage based protection at PCC to trip due to a voltage regulation issue not related to actual faults in the system. there might also be complications detecting phase to ground faults as earthing connection and transformer connection can prevent the flow of zero sequence fault current flow at the PCC.
A distance protection based solution is proposed in \cite{PR140}. Distance protection is directional, and this property makes it suitable for non-radial networks. The authors in \cite{PR140} propose connecting distance protection on the low voltage side of the distribution transformer to avoid the cost of a voltage transformer. To tackle the issue of no measurable zero sequence current on the low voltage side due to a fault on the high voltage side of the transformer, the scheme proposes using a preset estimate of the impedance behind the distance relay during phase-to-earth faults. Although this research in \cite{PR140} shows good potential for distance protection methods, more research is needed to properly determine the preset estimates.
The authors in \cite{PR140} propose another OC protection scheme with the help of a central controller. This scheme is only viable in systems that do not have inverter interfaced DER.
A protection scheme for inverter interfaced DER is proposed in \cite{PR142}. The authors in \cite{PR142} use the sequence currents to detect faults. Then it uses time delays to achieve protection coordination. The implementation of this method requires a fault current limiter. The drawbacks of the proposed method are discrimination capabilities. The zero sequence current during a high impedance fault might be similar to unbalanced load current. The scheme also does not take into account the possibility of transformer configuration and earthing paths may provide multiple or no zero sequence current paths.
The authors in \cite{PR143} propose a negative sequence current based protection scheme. And for high impedance fault, the method proposed in \cite{PR144} is incorporated. This method uses blocking signals to ensure selectivity in fault and it also provides backup protection for circuit breakers. The proposed method performs optimally with proper communication. But it can also perform sub optimally if there is a communication failure. The tradeoff would be that if there is a communication failure, the method will disconnect more consumers than necessary. 
A differential energy based scheme is presented in  \cite{PR145}. The current at each protection device is converted to spectral energy content using S-transform. The spectral energy using the data gathered at the opposite end is then calculated. These results are used to calculate the differential spectral energy. This proposed scheme is reliant on communication and there is no backup suggested in case of communication failure. 
Another protection scheme based on differential current and symmetrical current components is proposed in \cite{PR146}. The protection scheme proposed is proven to effectively determine phase to earth and line to line faults. The authors propose the use of zero and negative sequence currents to identify the faults. They take into account the unbalanced loading conditions and coordinate protection settings for the relays. They are tuned in such a way that a tripping will not occur during the extremes of unbalanced conditions and a trip should occur if there is a high impedance fault in the relay’s protection zone.