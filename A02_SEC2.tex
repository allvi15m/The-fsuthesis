In this section the problem of optimizing the energy store based on current and forecusted data is formulated. Fig. \ref{fig:system_arch} depicts an example system. Here, the energy storage management system (ESMS) is in charge of controlling the battery connected to the grid with the objective of getting the most cost optimum use of the resources available. The objective of the ESMS is to optimize the use cost of energy storage under different pricing schemes by taking advantage of RTP or time of use (TOU) prices, load and DG generation forecasting.  Fig. \ref{fig:F1_CA} shows the top-level architecture of the ESMS. As seen in the figure, the inputs of the system are the real-time price (RTP) prediction, load prediction, and the DG output prediction, which in this case is a photovoltaic (PV) plant output. It will also consider the current state of the load, current PV generation, and the state of charge of the ES. The output of the ESMS is the optimum battery charge and discharge control references based on the current and forecasted data.

\begin{figure}[!htbp]
\centering
\includegraphics[width=0.6\linewidth]{figs/A8/System_architecture.png}
\caption{Example test system architecture}
\label{fig:system_arch}
\vspace{-3mm}
\end{figure}

%  Fig. \ref{fig:F1_CA} shows the top-level architecture of the ESMS. As seen in the figure, the inputs of the system are the real-time price (RTP) prediction, load prediction, and the DG prediction, which in this case is a photovoltaic (PV) plant. It will also consider the current state of the load, PV generation, and ES. The output of the ESMS is the optimum battery charge and discharge control references based on the current and forecasted data.

\begin{figure}[!ht]
    \centering
    \includegraphics[width = 0.8\linewidth]{figs/A8/EMS_FIG.png}
    \caption{Controller top level architecture}
    \label{fig:F1_CA}
\end{figure}

Determining the optimum reference for the ES in continuous domain is a difficult and computationally intensive tusk. To make the problem solvable is a fast manner the solution space of the problem is discretized. Fig. \ref{fig:F1_Dis} demonstrates an example of the discretized solution space. The horizontal axis of the figure represents time, and the vertical axis represents discrete steps in the state of charge (SOC) of the energy storage. The solution space in the figure assumes that forecast data for the next 45 minutes are available and a control action is taken every 15 minutes. The SOC of the ES is limited between 80\% and 20\%. It is also assumed that the ES can discharge a maximum of 40\% of its maximum SOC and charge a maximum of 20\% of its SOC in a 15 minutes. The space between the upper and lower bounds of the SOC is discretized in steps of 20\%. This allows us to define the solution space as a directional graph. In the graph represented in the figure a discrete value of SOC at a discrete time is a node or a vertex. The possible paths from one node to another node at the next  time step can be defined as a directional edge. Using these definitions the solution space can be defined as a directional graph. The parameters chosen for this simple example are arbitrary. A similar graph can be constructed for various situations based on system properties and constraints.

% In order to find the optimum cost solution based on the current status of the system and future forecasts, the optimization problem is formulated using a graph search problem approach. To represent the solution space of the problem as a graph, the state of charge (SOC) of the ES, and both the prediction horizon and the control horizon are discretized. Fig. \ref{fig:F1_Dis} demonstrates an example of the discretized solution space.
\begin{figure}[!ht]
    \centering
    \includegraphics[width = 0.6\linewidth]{figs/A8/F1_1_Dis.png}
    \caption{Discretizing solution space}
    \label{fig:F1_Dis}
\end{figure}

% The horizontal axis of the figure represents time, and the vertical axis represents discrete steps in the state of charge (SOC) of the energy storage. In this simple example scenario, it is assumed that the algorithm recalculates the solution every 15 minutes (control horizon) based on available data. The SOC of the energy storage system (ESS) is discretized in steps of 20\%, and the SOC is limited between 80\% and 20\%. It is also assumed that the ESS can discharge a maximum of 40\% of its maximum SOC and charge a maximum of 20\% of its SOC in a 15 minute time step. These values are chosen arbitrarily in this simple example scenario to explain the problem. Taking these features into consideration, a directed graph is constructed looking ahead three-time steps into the future. The square boxes represent nodes on the graph. The numbers inside the boxes represent the SOC of the ESS at that node. The arrows from the boxes represent all the possible states the ESS can be in the next time step according to the constraints of the system. The arrows are treated as edges of the graph. In this case, the edges are unidirectional. The goal is to find the most cost-efficient path to reach T = 45 minutes. Although in this example the algorithm considers T = 45 as the final stage, in actual application the final stage can be determined based on the actual use case.